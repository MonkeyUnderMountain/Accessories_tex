% packages_pdflatex.tex


%--------------------------------------------------------------------------
% 核心包加载(适用于 pdflatex)
%--------------------------------------------------------------------------
% 基础功能包
\usepackage{graphicx}                             % 图形插入支持
\usepackage{longtable}                            % 长表格支持
\usepackage{amsmath,amsthm,amssymb,amsfonts}      % 数学符号和字体支持
\usepackage{thmtools}                             % 高级定理工具包(提供declaretheorem)
\usepackage[dvipsnames]{xcolor}                   % 颜色支持
\usepackage{tikz}                                 % 矢量绘图
\usepackage{pstricks-add}                         % PSTricks扩展
\usepackage{tikz-cd}                              % 交换图绘制
\usepackage[all]{xy}                              % 交换图绘制
\usepackage{mathtools}                            % 数学工具包(提供多种数学环境和命令)

% 页面设置和布局
\usepackage{geometry}
\geometry{
  a4paper,
  top=2.5cm,
  bottom=2.5cm,
  left=2.5cm,
  right=2.5cm,
}
\renewcommand{\baselinestretch}{1.3}

% 交互功能
\usepackage[
  colorlinks,
  linkcolor=blue,
  urlcolor=magenta,
  citecolor=cyan
]{hyperref}       % 超链接支持
\usepackage[nameinlink]{cleveref} % 智能引用(自动添加前缀)

% 列表环境标号配置
\usepackage{enumitem}       % Required for custom labels
\setlist[enumerate,1]{label=(\alph*)} % 第一层级:小写字母加括号 (a), (b), ...)
\setlist[enumerate,2]{label=(\roman*)} % 第二层级:小写罗马数字加括号 (i), (ii), ...)

%--------------------------------------------------------------------------
% 定理环境可视化配置
%--------------------------------------------------------------------------
% 定义具体定理环境(共享计数器)
\theoremstyle{definition}
  \newtheorem{definition}{Definition}[section]
  \declaretheorem[sibling=definition,name=Proposition]{proposition}
  \declaretheorem[sibling=definition,name=Theorem]{theorem}
  \declaretheorem[sibling=definition,name=Lemma]{lemma}
  \declaretheorem[sibling=definition,name=Corollary]{corollary}
  \declaretheorem[sibling=definition,name=Conjecture]{conjecture}
  \declaretheorem[sibling=definition,name=Question]{question}
  \declaretheorem[sibling=definition,name=Claim]{claim}
\theoremstyle{remark}
  \declaretheorem[sibling=definition,name=Remark]{remark}
  \declaretheorem[sibling=definition,name=Example]{example}

% 定义step子环境(在proof环境中使用,每个proof重置计数器)
\newcounter{step}
\newcounter{globalstep} % Global unique step counter for anchors
\newenvironment{step}[1][]{%
  \refstepcounter{step}%
  \refstepcounter{globalstep}% This creates unique anchors
  \par\smallskip%
  \noindent\textbf{Step \thestep\if\relax\detokenize{#1}\relax\else\ (#1)\fi.}\ %
}{%
  \par\smallskip%
}
\renewcommand{\theglobalstep}{\arabic{step}}

% 定义case子环境(在proof环境中使用,每个proof重置计数器)
\newcounter{case}
\newcounter{globalcase} % Global unique case counter for anchors
\newenvironment{case}[1][]{%
  \refstepcounter{case}%
  \refstepcounter{globalcase}% This creates unique anchors
  \par\smallskip%
  \noindent\textbf{Case \thecase\if\relax\detokenize{#1}\relax\else\ (#1)\fi.}\ %
}{%
  \par\smallskip%
}
\renewcommand{\theglobalcase}{\arabic{case}}

\AtBeginEnvironment{proof}{%
  \setcounter{case}{0}%
  \setcounter{step}{0}%
}

% 设置cleveref引用格式
\crefname{definition}{Definition}{Definitions}
\crefname{proposition}{Proposition}{Propositions}
\crefname{theorem}{Theorem}{Theorems}
\crefname{lemma}{Lemma}{Lemmas}
\crefname{corollary}{Corollary}{Corollaries}
\crefname{conjecture}{Conjecture}{Conjectures}
\crefname{question}{Question}{Questions}
\crefname{remark}{Remark}{Remarks}
\crefname{example}{Example}{Examples}
\crefname{proof}{Proof}{Proofs}
\crefname{claim}{Claim}{Claims}
\crefname{globalstep}{Step}{Steps} % Global unique step counter
\crefname{globalcase}{Case}{Cases} % Global unique case counter
\crefname{section}{Section}{Sections}
\crefname{subsection}{Subsection}{Subsections}